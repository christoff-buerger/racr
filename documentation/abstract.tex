\cleardoublepage\thispagestyle{empty}
\begin{center}{\bfseries Abstract}\end{center}
\begin{onehalfspace}
This report presents RACR, a reference attribute grammar library for the programming language Scheme.

\vspace{7pt}

RACR supports incremental attribute evaluation in the presence of abstract syntax tree rewrites. It
provides a set of functions that can be used to specify abstract syntax tree schemes and their
attribution and construct respective trees, query their attributes and node information and annotate
and rewrite them. Thereby, both, reference attribute grammars and rewriting, are seamlessly integrated,
such that rewrites can reuse attributes and attribute values change depending on performed rewrites –
a technique we call Reference Attribute Grammar Controlled Rewriting. To reevaluate attributes
influenced by abstract syntax tree rewrites, a demand-driven, incremental evaluation strategy, which
incorporates the actual execution paths selected at runtime for control-flows within attribute
equations, is used. To realize this strategy, a dynamic attribute dependency graph is constructed
throughout attribute evaluation – a technique we call Dynamic Attribute Dependency Analyses.

\vspace{7pt}

The report illustrates RACR's motivation, features, instantiation and usage. In particular its
application programming interface is documented and exemplified. The report is a reference manual for
RACR developers. Further, it presents RACR’s complete implementation and therefore provides a good
foundation for readers interested into the details of reference attribute grammar controlled rewriting
and dynamic attribute dependency analyses.
\end{onehalfspace}
\vfill\null