% This program and the accompanying materials are made available under the
% terms of the MIT license (X11 license) which accompanies this distribution.

% author: C. Bürger

\cleardoublepage\thispagestyle{empty}
\begin{center}{\bfseries Abstract}\end{center}
\begin{onehalfspace}
This report presents \emph{RACR}, a reference attribute grammar library for the programming language \emph{Scheme}.

\vspace{7pt}

\emph{RACR} supports incremental attribute evaluation in the presence of arbitrary abstract syntax tree rewrites. It provides a set of functions that can be used to specify abstract syntax tree schemes and their attribution and construct respective trees, query their attributes and node information and annotate and rewrite them. Thereby, both, reference attribute grammars and rewriting, are seamlessly integrated, such that rewrites can reuse attributes and attribute values change depending on performed rewrites – a technique we call Reference Attribute Grammar Controlled Rewriting. To reevaluate attributes influenced by abstract syntax tree rewrites, a demand-driven, incremental evaluation strategy, which incorporates the actual execution paths selected at runtime for control-flows within attribute equations, is used. To realise this strategy, a dynamic attribute dependency graph is constructed throughout attribute evaluation – a technique we call Dynamic Attribute Dependency Analyses.

\vspace{7pt}

Besides synthesised and inherited attributes, \emph{RACR} supports reference, parameterised and circular attributes, attribute broadcasting and abstract syntax tree and attribute inheritance. \emph{RACR} also supports graph pattern matching to ease the specification of complex rewires, whereas patterns can reuse attributes for rewrite conditions such that complex analyses that control rewriting can be specified. Similarly to attribute values, tests for pattern matches are automatically cached. Further, linear pattern matching complexity is guaranteed, if all attributes used in patterns have linear complexity. Thus, the main drawback of graph rewriting, the matching problem of polynomial complexity for bounded pattern sizes, is attenuated.

\vspace{7pt}

The report illustrates \emph{RACR's} motivation, features, instantiation and usage. In particular its
application programming interface is documented and exemplified. The report is a reference manual for
\emph{RACR} developers. Further, it presents \emph{RACR’s} complete implementation and therefore provides a good
foundation for readers interested into the details of reference attribute grammar controlled rewriting
and dynamic attribute dependency analyses.
\end{onehalfspace}
\vfill\null